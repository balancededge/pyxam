\documentclass[9pt]{extarticle}
\usepackage[letterpaper, landscape, margin=.5in]{geometry}
\usepackage{multicol}
\usepackage{hyperref}
\pagestyle{empty}
\begin{document}
\centering\section*{Pyxam Cheat Sheet}
\begin{multicols}{2}
\raggedright\subsection*{Running Pyxam}
Usage pyxam.py [Options] template \\
{\bf Command list:} \\
\begin{tabular}{l l l l}
  out & -o  & [file] & Set the output directory \\ 
  temp & -tmp & [file] & Set the temporary directory \\
  figure & -fig & [file] & Set the figure directory \\
  number & -n & [int] & Set the number of exams to generate \\
  sample & -smp & [int] & Set the default sample number \\
  recompilation & -r & [int] & Set the number of recompilations \\
  title & -t & [title] & Set the exam name \\
  format & -f & [format] & Set the export format (tex, pdf, dvi, html) \\
  shell & -shl & [shell] & Set the shell (python, matlab, octave, julia) \\
  method & -m & [method] & Set the selection method (random, sequence) \\
  population & -p & [file] & Set the class list for mixing \\
  solutions & -s & & Enable solutions \\
  alphabetize & -a & & Enable lettered versioning rather than numbered \\
  clean & -c & & Enable LaTeX cleanup \\
  interactive & -i & & Enable interactive pdflatex \\
  logging & -l & & Enable logging \\
  debug & -d & & Disable file cleanup (keep temporary files) \\
\end{tabular} 

For more details see README.md \\
\subsection*{LaTeX Commands}

\begin{tabular}{l l}
{\bf Python:} & \\
\verb \ \verb Pexpr{...} & Run Python code snippet and print the result \\
\verb \ \verb Pexprs{...} & Run Python code snippet silently \\
\verb \ \verb Pverb{...} & Run Python code and print the code in a verbatim block \\ 
\verb \ \verb Pblock{...} & Run Python code silently \\
\verb \ \verb Pfig[caption]{...} & Display a figure with the specified caption \\
{\bf Importing Questions:} & \\
\verb \ \verb Pimport{file} & Import a single question file \\
\verb \ \verb Pimport{file1|file2} & Import either file1 or file2 \\
\verb \ \verb Pimport[n]{file} & Import a single question file n times \\
\verb \ \verb Pimport{dir} & 
Import a single question from directory dir \\ 
\verb \ \verb Pimport[n]{dir1|dir2|dir3} &
Import n questions from dir1, dir2, or dir3 \\ \\
{\bf Constants:} & \\
\verb \ \verb Pconst{VERSION} & Get the exam version number or letter \\
\verb \ \verb Pconst{TITLE} & Get the exam title \\
\verb \ \verb Pconst{STUDENT} & Get a student's name \\
\verb \ \verb Pconst{STUDNUM} & Get a student's number \\ \\
{\bf Options:} & \\
\verb \ \verb Parg{args} & Equivalent to running pyxam.py [args] template
\end{tabular}
\subsection*{Examples}
\begin{description}
    \item See {\it examples/template.tex} for a simple exam that implements all of Pyxam's features 
    \item See {\it examples/exam.tex} for examples of more complex problems 
    \item See {\it examples/github.tex} for an introductory guide to github
    \item See {\it README.md} for a general overview of the tools and basic usage 
\end{description}
\subsection*{Development Tools}
\begin{description}
    \item {\bf Github} \\
        The version control system used for Pyxam. Github allows for easy
        management and access of source code. Github can be found at
        \url{https://github.com/ and the project page} for Pyxam
        can be found at \url{https://github.com/balancededge/pyxam}.  
    \item {\bf Git-Cola} \\
        A GUI client for github on Unix systems. A convenient tool when
        working with a larger number of files in sub directories where the
        command line may be less suitable. Git-Cola can be installed through
        Yast.
    \item {\bf PyCharm} \\
        A Python IDE with all the bells and whistles. PyCharm makes
        programming Python easy and enjoyable whilst also still being one of the
        most responsive editors available. PyCharm Community edition is free
        and can be found at \url{https://www.jetbrains.com/pycharm/}.
    \item {\bf Dillinger.io} \\
        A Markdown browser based editor. Dillinger is simple and elegant a
        great solution for writing Markdown documents. Dillinger can be found
        at \url{http://dillinger.io/}.
    \item {\bf Emacs } \\
        A powerful and configurable text editor. An ideal environment when
        working in a large variety of programming languages and with a large
        number of file formats. Emacs can be installed through Yast.
\end{description}
\subsection*{Emacs Shortlist}
\begin{tabular}{l l}
M-p & Previous shell command \\
C-x-C-v RET & Refresh the currently selected buffer \\
C-x-1 & Close all windows except the currently selected one \\
C-x-2 & Split window vertically \\
C-x-3 & Split window horizontally \\
C-x-0 & Close the currently selected window \\
C-x-k RET & Kill the currently selected buffer \\
\end{tabular}

\end{multicols}
\end{document}
