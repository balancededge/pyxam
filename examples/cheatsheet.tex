
\documentclass[9pt]{extarticle}
\usepackage[letterpaper, landscape, margin=.5in]{geometry}
\usepackage{multicol}
\usepackage{hyperref}
\pagestyle{empty}
\begin{document}
\centering\section*{Pyxam Cheat Sheet v0.3.2}
\begin{multicols}{2}
\raggedright\subsection*{Running Pyxam}
Usage pyxam.py [Options] template \\
{\bf Command list:} \\
\begin{tabular}{l l l l}
alphabetize & -a &  & Enable lettered versioning\\
out & -o & [out] & Set the output directory\\
recomps & -r & [recomps] & The number of LaTeX recompilations\\
version & -v &  & Show the version number\\
population & -p & [population] & The class list CSV\\
commands & -cmd &  & Display all available commands\\
method & -m & [method] & The selection method for CSVs\\
readme & -rd &  & Generate a new README.md file\\
number & -n & [number] & Set the number of exams to generate\\
help & -h &  & Show a help message\\
list & -ls &  & List all available formats\\
random & -rng & [random] & Set the seed for rng\\
figure & -fig & [figure] & Set the figure directory\\
plugins & -plg &  & List all currently loaded plugins\\
logging & -l & [logging] & Set the logging level for pyxam\\
 & & & 10: DEBUG\\
 & & & 20: INFO\\
 & & & 30: WARNING\\
 & & & 50: CRITICAL\\
solutions & -s &  & Enable soultions\\
format & -f & [format] & The export format\\
api & -api &  & Run Pyxam in api mode\\
cheatsheat & -cht &  & Generate a new cheatsheet\\
tmp & -tmp & [tmp] & Set the temporary directory\\
debug & -d &  & Disable file cleanup\\
title & -t & [title] & The title of the exam\\
shell & -shl & [shell] & The shell used to weave the exam
\end{tabular} 

For more details see README.md \\
\subsection*{Pyxam Bang Commands}

\begin{tabular}{l l}
pyxam! sexpr & 
    Run a python expression silently
    \\
pyxam! verb & 
    Run a python block and return a verbatim copy of the code
    \\
pyxam! expr & 
    Run a python expression and echo the result
    \\
pyxam! fig & 
    Insert a python figure
    \\
pyxam! studentname & None\\
pyxam! studentnumber & None\\
pyxam! args & 
    Load args as though they were command line options.
    \\
pyxam! block & 
    Run a python block silently
    \\
pyxam! import & 
    Insert a question
    \\
pyxam! def & 
    Define a constant
    
\end{tabular}
\subsection*{Examples}
\begin{description}
    \item See {\it examples/template.tex} for a simple exam that implements all of Pyxam's features 
    \item See {\it examples/exam.tex} for examples of more complex problems 
    \item See {\it examples/github.tex} for an introductory guide to github
    \item See {\it README.md} for a general overview of the tools and basic usage 
\end{description}
\subsection*{Development Tools}
\begin{description}
    \item {\bf Github} \\
        The version control system used for Pyxam. Github allows for easy
        management and access of source code. Github can be found at
        \url{https://github.com/ and the project page} for Pyxam
        can be found at \url{https://github.com/balancededge/pyxam}.  
    \item {\bf Git-Cola} \\
        A GUI client for github on Unix systems. A convenient tool when
        working with a larger number of files in sub directories where the
        command line may be less suitable. Git-Cola can be installed through
        Yast.
    \item {\bf PyCharm} \\
        A Python IDE with all the bells and whistles. PyCharm makes
        programming Python easy and enjoyable whilst also still being one of the
        most responsive editors available. PyCharm Community edition is free
        and can be found at \url{https://www.jetbrains.com/pycharm/}.
    \item {\bf Dillinger.io} \\
        A Markdown browser based editor. Dillinge   r is simple and elegant a
        great solution for writing Markdown documents. Dillinger can be found
        at \url{http://dillinger.io/}.
    \item {\bf Emacs } \\
        A powerful and configurable text editor. An ideal environment when
        working in a large variety of programming languages and with a large
        number of file formats. Emacs can be installed through Yast.
\end{description}
\subsection*{Emacs Shortlist}
\begin{tabular}{l l}
M-p & Previous shell command \\
C-x-C-v RET & Refresh the currently selected buffer \\
C-x-1 & Close all windows except the currently selected one \\
C-x-2 & Split window vertically \\
C-x-3 & Split window horizontally \\
C-x-0 & Close the currently selected window \\
C-x-k RET & Kill the currently selected buffer \\
\end{tabular}

\end{multicols}
\end{document}
