% Adopted from octave_sample.mdw from http://mpastell.com/pweave/examples/
% Compile this file to pdf using the command:
%   $ pyxam -f pdf -shl matlab matlab_example.tex
% Compile this file to moodle using the command:
%   $ pyxam -f moodle -shl matlab matlab_example.tex
% Compile this file to html using the command:
%   $ pyxam -f html -shl matlab matlab_example.tex
% Set the title of the output document to matlab_example
<%
pyxam.args('-t matlab_example')
%>
\documentclass[12pt]{exam}

    % The graphicx package is needed for figures and images
    \usepackage[pdftex]{graphicx}

    \begin{document}
        \section*{Using Octave with Pweave}
            You can also use Pyxam with GNU Octave or Matlab. You can use inline code chunks like in Python documents: \\
            Give y value 300 <%y=300;%> in hidden chunk. \\
            And let's verify that it worked: \\
<<>>=
y
@
            You can also display the result from inline chunk 2+5=<%=2+5%>
        \section*{Solving Least Squares}
            Trying out plotting features:
<<fig=True>>=
x = (0:25) + randn(1, 26);
y = linspace(0, 5, length(x));
a = x' \ y'
plot(x , y, 'o')
hold on
plot(x, a*x, 'r')
hold off
figure()
hist(a*x - y)
title('Histogram of residuals')
@
            And include a plot but hide the code:
<<fig=True, echo=False, caption="Sinc function">>=
x = linspace(0, 4*pi, 200);
plot(x, sinc(x), 'linewidth', 1)
hold on
plot(x, sinc(0.7*x), 'g', 'linewidth', 1)
hold off
xlabel('x')
ylabel('sinc(x)')
@
