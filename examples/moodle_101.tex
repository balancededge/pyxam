% Compile this document to a pdf using the command:
%   $ pyxam -s -f pdf moodle_101.tex
% Compile this document to html using the command:
%   $ pyxam -s -f html moodle_101.tex
% Set the title of the output document to moodle_101
<%
pyxam.args('-t moodle_101')
%>
\documentclass[12pt]{exam}
\usepackage[pdftex]{graphicx}
\begin{document}
\section*{Moodle 101}
Moodle has a much smaller set of options than other formats as you are not only displaying questions, but also
formulating a model for solving them. Basic questions in Moodle behave almost exactly like those of other formats. This
document will explain the more complex situations in Moodle.
\subsection*{Code Blocks in Moodle}
For most use cases there are five main ways to wrap code,
\begin{description}
\item Code chunk for a figure
\item An echoed Code Chunk
\item A silent Code Chunk
\item An echoed inline block
\item A silent inline block
\end{description}
Immediately there is confusion produced from the fact that echo here means two different things. For the code chunk
echo means that your code will be copied verbatim into the final document inside of the format specific verbatim
section. For the inline block echo means that the result of the provided expression will be echoed (like what occurs
at the Python command line). As an example:
<<echo=True>>=
3 + 4
@
and <%= 3 + 4 %> produce very different results. You may have also noted that the echoed result of the inline block does not appear in
a verbatim section. This is very imporant to the Moodle format. Another detail of the inline echoed block is that it may
only contain a single expression unlike its silent equivalent.
\\\\
The next difference between chunks and blocks to be considered is how the print statement is treated by both. Here are
two examples of the two silent ways of writing code:
<<echo=False>>=
print('Hello World')
@
and <% print('Hello World') %>.
Both printed the resulting text however the chunk still printed that text into a verbatim section.
\subsection*{Numerical Questions}
To create a numerical question create a typical shortanswer question, but in the solutions block instead of writing
out the answer call the following api function:
\begin{verbatim}
pyxam.numerical(solution=[solution], tolerance=[tolerance], percent=[percent])
\end{verbatim}
This function will use print to produce the needed XML for moodle. In order for this to work this XML cannot be
within a verbatim section, as such the function must be called within a silent inline block.
\\\\
The values that are accepted by the solution are either integers, floats, or Wildcards. Regardless of which you use this
question will always have the same answer when opened in Moodle.
\subsection*{Calculated Questions}
In order to have different questions for different students in Moodle you need to use Wildcards and the Calculated
question type. What is going on here is that Moodle creates a small database for variables. For instance {x} may have
the values 1, 2, 3, and 4. Whenever you open that question in Moodle it makes a random selection of one value for each
variable. It should be noted that the same item index is used for every variable picked. Meaning if the first value for
{x} is used than the first value of {y} is used. This will allow us to do questions of the pick from a list sort.
\\\\
Pyxam provides the wildcard class to help create these datasets. They can be created directly from a list of numbers or
by defining a maximum, minimum, and a number to generate. Here are two examples:
<<>>=
x = pyxam.wildcard(set=[1, 2, 3, 4])
y = pyxam.wildcard(min=1, max=10, n=4)
@
When wildcards are used outside of Moodle they will pick a value based off of the exam version. So if we try to print
these values we can see their internal dataset as well as their value current value given the exam version:
\begin{verbatim}
The data set for x is: <%= x.set %>
The current value of x is: <%= x %>
The data set for y is: <%= y.set %>
The current value of y is: <%= y %>
\end{verbatim}
Additionally wildcards have most of the basic mathematical operations defined for
them, so you can add/subtract/divide/multiply/modulo etc. them together. Additionally you can add integers and floats to
them so long as the wildcard is the first variable in the operation.
\\
To make reference to your variables in a Moodle question you simply place them in curly brackets in an echoed inline
block. For example x is <%= {x} %>. When exporting to formats other than Moodle this block will take on the value of
x, but in Moodle it will be formatted so that Moodle recognizes it as a data item.
\\\\
The final step in using calculated questions is writing the solution. Like with numerical questions you can define
tolerances and the function call must occur in a silent inline block, however the solution argument is replaced with the
equation argument. This equation is a string that is used by Moodle to calculate the answer, hence the name
"calculated". For simple questions Moodle may provide the sufficient math operations to solve the equation, if for
instance if we wanted to add x and y, for the solution we would simply put:
\begin{verbatim}
pyxam.calculated(equation='{x} + {y}')
\end{verbatim}In which case the answer to the question will be calculated by Moodle every time you open the question. If however you
want to do more complex calculations than Moodle allows you can take an alternative approach. Calculate your result and
store it into a variable:
<<>>=
c = x ** y
@
Now c is your solution and for your equation you can simply put:
\begin{verbatim}
pyxam.calculated(equation='{c}')
\end{verbatim}This functionality essentially simulates a pick from a list question type.
\end{document}