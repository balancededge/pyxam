% Compile this document to pdf using the command:
%   $ pyxam -f pdf noweb_examples.tex
% Compile this document to html using the command:
%   $ pyxam -f html noweb_examples.tex
% Set the title of the output document to noweb_example
<%
pyxam.args('-t noweb_example')
%>
\documentclass[12pt]{exam}

 % The graphicx package is needed for figures and images
\usepackage[pdftex]{graphicx}

\begin{document}
    \section{noweb}
        Noweb is a literate programming tool for interweaving code and documentation. Usages come in two major flavors,
        chunks and inline blocks. Chunks allow for flags to be set and greater control over what happens to the embedded
        code. Inline blocks allow for code to be quickly and easily used directly within documentation.
        \subsection{chunks}
            Chunks are started with a set of double arrows folowed by an equal sign. Chunk options are set between the
            arrows. A chunk is closed by the @ sign. Below is an example of a standard chunk. By default chunks like
            these will be echoed as verbatim code in the final document.
<<>>=
print('Hello World')
@
            In order to stop the code from being echoed to the document the echo option can be disabled. In this case
            only text directly printed by the block will appear in the final document.
<<echo=False>>=
print('Hello World')
@
            Displaying figures can be done by setting the figure flag.
<<echo=False, fig=True>>=
import matplotlib.pyplot
matplotlib.pyplot.plot([1, 2, 3, 4])
matplotlib.pyplot.show()
@
            For additional options supported by Pyxam check out the documentation for Pweave at
            http://mpastell.com/pweave/chunks.html
        \subsection{inline}
            Inline code provides less options but follows a simpler syntax. It can be used in two ways. You can
            evaluate code silently like in the example below. While printed text from the code will still appear in the
            final document, it will no longer appear in a verbatim block.
<%
x = 'per' + 'cent'
%>
            Alternatively you can have the result of the code echoed directly into the document by adding the
            <%= x %> sign.
\end{document}
    
