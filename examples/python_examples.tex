% Compile this document to a pdf using the command:
%   $ pyxam -s -f pdf python_examples.tex
% Compile this document to html using the command:
%   $ pyxam -s -f html python_examples.tex
% Set the title of the output document to python_examples
<%
pyxam.args('-t python_examples')
%>
\documentclass[12pt]{exam}
\usepackage[pdftex]{graphicx}
\begin{document}
\section*{A Brief Overview of Python}
Python is a very high level dynamic programming language. Python emphasizes ease of development over performance and low
level features. These features can still be accessed however thanks to easy interfacing with c code and c based
libraries.
\subsection*{Hello World}
Python's hello world program is this simple:
<<echo=True>>=
print('Hello World')
@

\subsection*{Formatting Strings}
A simple task that may be convenient for producing strings is formatting. Here are a couple simple examples.
<<echo=True>>=
print('Today is {}'.format('Tuesday'))
print('The time is {hours}:{minutes}'.format(hours=4, minutes=55))
@

As you can see arguments for formatting can be supplied in order or with keyword arguments. Additional formatting
options can be found [online](https://docs.python.org/3.4/library/string.html).
\subsection*{Control Flow}
Control flow in python is handled with formatting. A tab (typically 3 or 4 spaces and not the tab character) denotes a
new code block similar to how curly braces denote a code block in c. The example below demonstrates a simple if
statement.
<<echo=True>>=
tired = True
late = False
if tired and late:
    print('sleep')
else:
    print('work')
@

\subsection*{Lists and Looping}
There are many iterable types in Python, but the simplest to understand is the list. A list is an ordered set of
objects. The objects can be of any type and the list can have items added or removed. Lists can be looped through
directly using a for loop or indirectly like you would in c. Python has other great features for working with iterable
objects such as the ability to use negative indexes to access the end of a list.
<<echo=True>>=
languages = ['python', 'lisp', 'ruby', 'c', 'c++']
print(languages)
languages.append('fortran')
print(languages)
languages.remove('c++')
print(languages)
# Directly looping through a list
for language in languages:
    print(language)
# Indirectly looping through a list backwards
for i in range(len(languages)):
    print(languages[-(i + 1)])
@

\subsection*{Dictionaries}
Dictionaries are one of the most versatile types in python. They are similar to maps in other programming languages.
They allow you to associate a key with a value. Keys can be any immutable values (meaning numbers, strings, and tuples)
and values can be anything. A list comprehension is used below. A list comprehension uses the following syntax:
\begin{verbatim}
[f(a_i) for a_i in some_iterable]
produces the list
[f(a_0), f(a_1), ..., f(a_n)]
where a is a member of some_iterable
\end{verbatim}
<<echo=True>>=
dictionary = {
    'hello': 'bonjour',
    'my': 'je',
    'name': 'nom',
    'is': 'est',
    'eric': 'eric'
}
english = 'Hello my name is Eric'
print(english)
translated = ' '.join(dictionary[word.lower()] for word in english.split(' '))
print(translated.capitalize())
@

\subsection*{Functions}
You can define functions in python that have their own scope. Python functions are first class objects meaning they can
be passed around just like any other object. Python also allows for lambda expressions.
<<echo=True>>=
global_flag = True
def test_flag():
    global global_flag
    return global_flag
print(test_flag())
test_flag = lambda : False
print(test_flag())
@

\subsection*{Regular Expressions}
To use regular expressions import the re module. To save yourself from writing many backslashes start patterns with the
letter r.
<<echo=True>>=
import re

unformatted = '7807478238'
print(unformatted)
formatted = re.sub(r'(\d{3})(\d{3})(\d{4})', r'(\1) \2-\3', unformatted)
print(formatted)
@

\subsection*{Math With Python}
Python supports most of the expected operations (bitwise and boolean operations not shown).
<<echo=True>>=
a = -4
b = 3.14
print('a =', a, 'b =', b)
c = a + b
print('a + b =', c)
c = a - b
print('a - b =', c)
c = a * b
print('a * b =', c)
c = a ** b
print('a to the power of b =', c)
c = a / b
print('a divided by b =', c)
c = a // b
print('a integer divided by b =', c)
c = a % b
print('a modulo b =', c)
c = abs(a)
print('|a| =', c)
c = int(b)
print('floor(b) =', c)
c = float(a)
print('float(a) =', c)
c = complex(a, b)
print('a + bi =', c)
c = c.conjugate()
print('c* =', c)
@

\subsection*{Math With Wildcards}
Wildcards are a utility class made to help with generating exams. They support most of the same operations that ints or
floats do. There are a few major exceptions. You cannot test Wildcards, for instance:
\begin{verbatim}
if x > 3:
\end{verbatim}
Works fine for ints or floats, but will fail for Wildcards as their value is not decided yet. The second limitation on
Wildcards is their ability to interact with ints or floats. They can be operate on ints or flaots, but cannot be
operated on themselves. For instance:
\begin{verbatim}
x + 3
\end{verbatim}
Is acceptable even if x is a wildcard. The result of the expression will be a new Wildcard. This though:
\begin{verbatim}
3 + x
\end{verbatim}
Will throw an Exception. The final limitations comes to complex numbers. Operations that would produce complex numbers
(square root of a negative) will still produce complex Wildcards however the complex and conjugate functions are not
supported for Wildcards generally.
<<echo=True>>=
import random

a = Wildcard(set=random.sample(range(1, 100), 10))
b = Wildcard(set=random.sample(range(1, 100), 10))
print('a =', a, 'b =', b)
c = a + b
print('a + b =', c)
c = a - b
print('a - b =', c)
c = a * b
print('a * b =', c)
c = a ** b
print('a to the power of b =', c)
c = a / b
print('a divided by b =', c)
c = a // b
print('a integer divided by b =', c)
c = a % b
print('a modulo b =', c)
c = abs(a)
print('|a| =', c)
c = int(b)
print('floor(b) =', c)
c = float(a)
print('float(a) =', c)
@

\subsection*{Figures}
Figures must be specified using the noweb chunk. There are wide variety of different plots
[here](http://matplotlib.org/1.4.1/examples/index.html).
<<echo=True>>=
from mpl_toolkits.mplot3d import axes3d
import matplotlib.pyplot as plt
from matplotlib import cm

fig = plt.figure()
ax = fig.gca(projection='3d')
X, Y, Z = axes3d.get_test_data(0.05)
cset = ax.contour(X, Y, Z, extend3d=True, cmap=cm.coolwarm)
ax.clabel(cset, fontsize=9, inline=1)
@
\subsection*{Numpy}
[What is Numpy?](http://numpy.readthedocs.org/en/latest/user/whatisnumpy.html)

NumPy is the fundamental package for scientific computing in Python. It is a Python library that provides a
multidimensional array object, various derived objects (such as masked arrays and matrices), and an assortment of
routines for fast operations on arrays, including mathematical, logical, shape manipulation, sorting, selecting, I/O,
discrete Fourier transforms, basic linear algebra, basic statistical operations, random simulation and much more.

Here are a couple of simple numpy examples. For further comments on these and more detailed examples head over
[here](https://docs.scipy.org/doc/numpy-dev/user/quickstart.html).
<<echo=True>>=
import numpy

# Declaring an integer array
a = numpy.array([2, 3, 4])
print(a)
print(a.dtype)
# Declaring a float array
b = numpy.array([1.2, 3.5, 5.1])
print(b)
print(b.dtype)
# Declaring a complex array
c = numpy.array([[1,2], [3,4]], dtype=complex)
print(c)
print(c.dtype)
# Creating a multi-dimensional array of zeros
d = numpy.zeros((3, 4))
print(d)
# Creating a multi-dimensional array of ones
e = numpy.ones((2,3,4))
print(e)
# Filling a range, arguments are min, max, step
f = numpy.arange(10, 30, 5)
print(f)
# Ranges can have float steps
g = numpy.arange(0, 2, 0.3)
print(g)
# Arrays can be reshaped
h = numpy.arange(10000).reshape(100, 100)
print(h)
# A set of basic operations
i = numpy.array([20, 30, 40, 50])
j = numpy.arange(4)
print(i - j)
print(j ** 2)
print(10 * numpy.sin(i))
print(a < 35)
@


\subsection*{CSV}
You can easily read in data from csv files and use them like any normal python lists. The with statement used below is a
context, it defines a start and end function, here it opens a filestream when the block begins and closes it when the
block finishes or fails.
<<echo=True>>=
import csv

with open('population.csv') as csvfile:
    reader = csv.reader(csvfile, delimiter=',', quotechar='"')
    for row in reader:
        print('\t'.join(row))
@

\end{document}