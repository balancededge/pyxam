% pyxam!docs This text is used by the python autodocs

\documentclass[12pt]{exam}
    % Enable graphics
    \usepackage[pdftex]{graphicx}

    \begin{document}

        \begin{center}
            \textbf{Example Questions}
        \end{center}

        \textbf{Name:} % pyxam!studentname

        \textbf{Student Number:} % pyxam!studentnumber

        \begin{questions}

            \titledquestion{1. Simple Essay Question}
                This is the simplest question possible. It is just a prompt and has no solution.

            \titledquestion{2. Simple Shortanswer Question}
                The simplest shortanswer questions are simply a prompt followed by a solution. When translating to moodle
                it is important to remember that the answer the student provides will have to match exactly in order to
                get the answer right. As an example, what is the capital of Alberta?
                \begin{solution}
                    Edmonton
                \end{solution}

            \titledquestion{3. Simple Numerical Question}
                Numerical questions work just like shortanswer questions only the answer should be in the form var = answer
                For example for the question what is $f(2)$ when $f(x) = x^2$?
                \begin{solution}
                    $f(2) = 4$
                \end{solution}

            \titledquestion{4. Numerical Question with Tolerance}
                Numerical questions can be given a tolerance using the pm command. When translating to moodle this
                will allow mark answers that fall within that range correct. For example the $y=\sqrt{2}$ would be?
                \begin{solution}
                    $y=1.414214 tolerance 10\%$
                \end{solution}

            \titledquestion{5. Simple Multiple Choice Question}
                Multiple choice questions are denoted by a the choices command. Any question that features choices
                will be converted to a multiple choice question. For example, who how old is Python?
                \begin{choices}
                    \choice 20 years
                    \choice 32 years
                    \CorrectChoice 25 years
                    \choice 17 years
                \end{choices}

            \titledquestion{6. Multi Select Question}
                When you want to have a multiple choice question with multiple answers that need to be selected in moodle
                simply add multiple correctchoice commands to the list of choices. For example which of the following
                are equivalent to 0?
                \begin{choices}
                    \CorrectChoice $x + x*-1$
                    \CorrectChoice $|\emptyset|$
                    \choice $\tau / \sqrt{\pi}$
                    \CorrectChoice $e^i\pi + 1$
                \end{choices}

            \titledquestion{7. Simple True False Question}
                True False questions are simply multichoice questions with a true or false correctchoice. These will be
                automatically converted to true false questions in Moodle.
                \begin{choices}
                    \CorrectChoice True
                    \choice False
                \end{choices}

            \titledquestion{8. Simple Calculated Question}
                Calculated questions can have randomly generated values assigned to parameters and a formula that is used
                to calculate the correct solution. Percent tolerances cannot be used with this question type. As a simple
                example what is ${x} + {y}$?
                \begin{solution}
                    ${x} + {y} = {x} + {y} tolerance 0.01$
                    where ${x} [1.0, 7.5, 6.0]$
                    where ${y} [2.5, 4.0, 9.8]$
                \end{solution}

            \titledquestion{9. Verbatim Python Code}
                You can display a snippet of python code verbatim by using the command pyxam!verb within a comment. Even
                though this code is being displayed it is still run. Here a python program that determines whether it is
                light out is shown.
                \begin{comment}
                    pyxam!verb

from time import localtime
# Is it dark outside?
dark = {
    1:16,2:17,3:18,4:19,5:19,6:20,7:20,8:19,9:18,10:17,11:16,12:16
}

light = {
    1:8,2:7,3:6,4:5,5:4,6:4,7:4,8:5,9:6,10:6,11:7,12:8
}
if localtime().tm_hour >=dark[localtime().tm_mon] or localtime().tm_hour < light[localtime().tm_mon]:
    answer = 'Yes'
else:
    answer = 'No'

                \end{comment}
                The answer to this question is the value of the variable answer. This is specified using the pyxam!expr
                command.
                \begin{solution}
                    % pyxam!expr answer
                \end{solution}

            \titledquestion{10 Using Python to Randomly Generate a Question}
                You can use python code to randomly generate a question. Python code in the block command will be run but not
                shown.
                \begin{comment} pyxam!block

import random

a = random.randint(0, 32)
b = random.randint(0, 32)

                \end{comment}
                You can then use those variables in your question. For example what is
                    % pyxam!expr a
                +
                    % pyxam!expr b
                ?
                \begin{solution}
                    $a + b =
                        % pyxam!expr a + b
                    $
                \end{solution}

            \titledquestion{11 Other Pyxam commands}
                There are other default commands available with pyxam. For instance the def command can be used to define constants.
                % pyxam!def date "Tuesday, March 8, 2016"
                Now everytime you use the command date you will get the result of the python code to the right of it. For example
                this standard was written on
                % pyxam!date
                . Def commands also support simple python expressions, for example you can find 2024 / 17
                % pyxam!def result 2024 / 17
                simply by defining the constant and than calling it as a command:
                % pyxam!result
                . The args command can be used to set command line arguments from a template file. The import  and fig commands
                are explored in later questions. The sexpr command can be used to run a python expression silently. The
                final two commands are the student\_name and student\_number commands. These will be replaced with blanks in
                if no csv file is provided but will otherwise be replaced with student information if available.

            \titledquestion{12 Question With a Figure}
                You can generate figures and use images in your questions using the fig command.
                \begin{comment} pyxam!fig

import matplotlib.pyplot as plt
plt.plot([1,2,3,4])
plt.ylabel('Y axis')
plt.xlabel('X axis')
plt.title('A Simple Graph')
plt.show()

                \end{comment}
                Your image will be displayed inline with the text and appear regardless of what output format you choose.

        \end{questions}



    \end{document}


