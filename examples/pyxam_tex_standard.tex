% To generate a pdf from this file with solutions: 
%    $ pyxam -s -f pdf pyxam_tex_standard.tex
\documentclass[12pt]{exam}

    % The graphicx package is needed for figures
    \usepackage[pdftex]{graphicx}

    \begin{document}
        \begin{questions}
            % Set the question's category to 'testing'
            <% pyxam.categorize(course='testing', category='standard') %>
            \titledquestion{1. Essay Question}
                This is the simplest question possible. It is just a prompt and has no solution.
            \titledquestion{2. Short Answer}
                Short answer questions are composed of a prompt and solution. When importing the question to Moodle the
                solution will be matched character for character.
                \begin{solution}
                    Place your solution in a solution block.
                \end{solution}
            \titledquestion{3. Multiple Choice}
                Multiple choice questions are made up of choices and correct choices. You can have as many of each as
                you need. All choices should appear in a choices block. By default multiple choice questions will be
                shuffled when imported to Moodle.
                \begin{choices}
                    \choice Text for choice A
                    \choice Text for choice B
                    \choice Text for choice C
                    \CorrectChoice Text for correct choice D
                \end{choices}
            \titledquestion{4. Shuffled Multiple Choice}
                Although multiple choice questions are automatically shuffled when added to Moodle they are not
                automatically shuffled when exporting to other formats such as pdf. To do this use the pyxam.shuffle
                function on your choices.
                \begin{choices}
                    <%= pyxam.shuffle(
                        '\\choice Text for choice A',
                        '\\choice Text for choice B',
                        '\\choice Text for choice C',
                        '\\CorrectChoice Text for correct choice D'
                    )%>
                \end{choices}
            \titledquestion{5. Multi Select}
                Multi select questions are a slight variation on multiple choice questions. By including multiple
                correct choices the Moodle format will automatically allow the user to select multiple answers.
                \begin{choices}
                    \choice Text for choice A
                    \CorrectChoice Text for correct choice B
                    \choice Text for choice C
                    \CorrectChoice Text for correct choice D
                \end{choices}
            \titledquestion{6. True or False}
                True or False questions are a variation of multiple choice questions where the only choices are True or
                False. Is $\sqrt{2}=7$?
                \begin{choices}
                    \choice True
                    \CorrectChoice False
                \end{choices}
            \titledquestion{7. Numerical Question}
                Numerical questions look like short answer questions except that in the solution block they call the
                pyxam.numerical function. By passing in a numerical solution the question will be set to the numerical
                type in moodle. What is $\sqrt{16}$?
                \begin{solution}
                    <%= pyxam.numerical(4) %>
                \end{solution}
            \titledquestion{8. Numerical Question With Absolute Tolerance}
                The pyxam.numerical function also supports a tolerance argument which allows for answers from $\pm$
                tolerance of the answer. What is $\sqrt{16}$ (a numerical tolerance of $\pm 0.1$ will be allowed)?
                \begin{solution}
                    <%= pyxam.numerical(4, tolerance=0.1) %>
                \end{solution}
            \titledquestion{9. Numerical Question With Percent Tolerance}
                Tolerance for numerical questions can also pe specified in percent by setting the percent flag. What is 
                $\sqrt{16}$ (a numerical tolerance of $\pm 1\%$ will be allowed)?
                \begin{solution}
                    <%= pyxam.numerical(4, tolerance=1, percent=True) %>
                \end{solution}
            \titledquestion{10. Using a Plot in a Question}
                Plots can be added to any question by setting the fig flag in the noweb format.
<<echo=False, fig=True>>=
import matplotlib.pyplot
# To add a dataset we use the pyplot module in matplotlib and provide a dataset to the plot function
matplotlib.pyplot.plot([1,2,3,4])
# The axis of the plot can be labeled using the ylabel and xlabel functions
matplotlib.pyplot.ylabel('Y axis')
matplotlib.pyplot.xlabel('X axis')
# A title can be added using the title function
matplotlib.pyplot.title('A Simple Graph')
# Use the show function to finalize the figure and display it in the question
matplotlib.pyplot.show()
@
            \titledquestion{11. Random Numbers}
                You can constructor a wildcard using pyxam.wildcard. Wildcards take a name and min max values. They will
                automatically generate random numbers for you. Wildcards support most basic arithmetic meaning they can
                be added together, multiply by eachother, etc. Operations between wildcards will produce another
                wildcard. This is important for the calculated question type. In order to include the values of
                wildcards in the question simply write them in a code snippet.
<%
# Set our parameters
a = pyxam.wildcard(min=0, max=10)
b = pyxam.wildcard(min=0, max=10)
%>
                Now that we've set up the question we can ask what is <%= a %> + <%= b %>?
                \begin{solution}
                    <% pyxam.numerical(a + b) %>
                \end{solution}
            \titledquestion{12. Plotting Random Numbers}
                Plots can use wildcards as their plotted data.
<<echo=False, fig=True>>=
import matplotlib.pyplot
# To add a dataset we use the pyplot module in matplotlib and provide a dataset to the plot function
a = pyxam.wildcard(set=[1, 2, 3, 4])
b = pyxam.wildcard(set=[5, 6, 7, 8])
c = pyxam.wildcard(set=[9, 0, 1, 2])
d = pyxam.wildcard(set=[3, 4, 5, 6])
matplotlib.pyplot.plot([a, b, c, d])
# The axis of the plot can be labeled using the ylabel and xlabel functions
matplotlib.pyplot.ylabel('Y axis')
matplotlib.pyplot.xlabel('X axis')
# A title can be added using the title function
matplotlib.pyplot.title('A Simple Graph')
# Use the show function to finalize the figure and display it in the question
matplotlib.pyplot.show()
@
            \titledquestion{13. Picking Parameters From a List}
                Wildcards can also be used to pick parameters from a list. Which parameter is picked will depend on the
                exam number, ie. version 1 of the exam will pick the first parameter, version 2 will pick the second,
                and so on. To create this type of wildcard simply specify the set argument.
<%
# Define the lists for our parameters
a = pyxam.wildcard(set=[1, 2, 3, 4])
b = pyxam.wildcard(set=[1, 2, 3, 4])
%>
                What is <%= a %> + <%= b %>?
                \begin{solution}
                    <% pyxam.numerical(a + b) %>
                \end{solution}
            \titledquestion{14. Calculated Question}
                Calculated questions can be used to create questions that randomly change when accessed on Moodle. When
                outputting to a format other than moodle these questions act just like numerical questions. You can
                use the n argument to the wildcard to set how many moodle values will be generated and the decimals
                argument to specify how many decimal points should be in the parameters.
<%
a = pyxam.wildcard(min=0, max=10, n=3, decimals=1)
b = pyxam.wildcard(set=[2.5, 4.0, 9.8])
%>
                To reference variables within your prompt you have to put the wildcards in curly brackets. So here we
                would ask what is <%= {a} %> + <%= {b} %>. Next call the pyxam.calculated function in the solution
                block. The first argument of the function must be the moodle equation used to calculate solutions.
                The tolerance and percent arguments fron numerical can also be used here. Finally call the pyxam.dataset
                function outside the solution block and pass in your wildcards.
                \begin{solution}
                    <% pyxam.calculated('{a} + {b}', tolerance=0.1) %>
                \end{solution}
                <% pyxam.dataset(a, b) %>
            \titledquestion{15. Calculated Question Picked From List}
                Just like with numerical questions you can also use the pick from a list with calculated questions and
                have the solution picked rather than calculated.
<%
a = pyxam.wildcard(set=[1, 2, 3, 4])
%>
                For example what is <%= {a} %>?
                \begin{solution}
                    <% pyxam.calculated('{a}') %>
                \end{solution}
                <% pyxam.dataset(a) %>
            \titledquestion{16. Constants}
                Constants can be referened by the following variables.
                <%= pyxam.number %> will be replaced with the exam number.
                <%= pyxam.version %> will be replaced with the exam version.
                <%= pyxam.student_first_name %> will be replaced with the student's first name if available
                <%= pyxam.student_last_name %> will be replaced with the student's last name if available
                <%= pyxam.student_name %> will be replaced with the student's full name if available.
                <%= pyxam.student_number %> will be replaced with the student number if available.
        \end{questions}
    \end{document}
