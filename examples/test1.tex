\documentclass{exam}
% from MECE615_2014_hmwrk1.tex
\usepackage{amsmath}
\printanswers
%\noprintanswers

\newcommand{\semyear}{2015}

%\documentclass[10pt]{article}
%%%%%%%%%%%%%%%%%%%%%%%%%%%%%%%%%%%%%%%%%%%%%%%%%%%%%%%%%%%%%%%%%%%%
%
%   ASSIGNMENT #1 FOR MECE 615 (Fall 2012)
%
%%%%%%%%%%%%%%%%%%%%%%%%%%%%%%%%%%%%%%%%%%%%%%%%%%%%%%%%%%%%%%%%%%%%%


\textheight22cm \textwidth18.5cm \topmargin-0.5cm
\oddsidemargin-1cm \evensidemargin-1mm
\parindent0mm
\parskip2.ex plus1ex minus1ex
%\pagestyle{empty}

%%%%%%%%%%%%%%%%%%%%
% SPECIAL SYMBOLS
%%%%%%%%%%%%%%%%%%%%
%\newcommand{\celsius}{\mbox{}\ensuremath{^{\circ}}\mbox{C}}
%\newcommand{\fahrenheit}{\mbox{}\ensuremath{^{\circ}}\mbox{F}}
%\newcommand{\WmmK}{\mbox{W/(m$^2$ K)}}
%\newcommand{\WmK}{\mbox{W/(m K)}}

%%%%%%%%%%%%%%%%%%%%
% EQUATION SPECIALS
%%%%%%%%%%%%%%%%%%%%
%\newcommand{\pdif}[2]{\frac{\partial #1}{\partial #2}}
%\newcommand{\psecdif}[2]{\frac{\partial^2 #1}{\partial #2^2}}
%\newcommand{\ddif}[2]{\frac{d #1}{d #2}}
%\newcommand{\gdif}[2]{\frac{D #1}{D #2}}
%%%%%%%%%%%%%%%%%%%%


\begin{document}

\lfoot{}
\cfoot{}
\rfoot[]{Page \thepage\ of \numpages}
\firstpageheader{}{}{}

\runningheader{MEC E 615}{Assignment Continued}{Winter \semyear}


%\runningfooter{}{Page \thepage\ of \numpages}{}
%\pagenumbering{arabic}

%%%%%%%%%%%%%%%%%%%%%%%%%%%%%%%%%%%%
%   HEADER
%%%%%%%%%%%%%%%%%%%%%%%%%%%%%%%%%%%%
\begin{tabular}{p{10.8cm}|l}
  & \large\bfseries\sffamily PDE Control\\
 \huge\bfseries\sffamily\centering ASSIGNMENT \#1 &
                      \large\bfseries\sffamily MEC E 615\\
  & \large\mdseries\rmfamily Winter \semyear\\
 \large\mdseries\rmfamily\centering  &
                      \large\mdseries\rmfamily Instructor: Dr. Bob Koch\\
\end{tabular}

%%%%%%%%%%%%%%%%%%%%%%%%%%%%%%%%%%%%
%   INSTRUCTIONS
%%%%%%%%%%%%%%%%%%%%%%%%%%%%%%%%%%%%
\begin{footnotesize}
\begin{tabular}{rp{13cm}}
 \textbf{Instructions:} --- & Do not hand in, solutions will be posted
% --- & .\\
% --- & Late assignments (after 13:00 on due date) will not be accepted.\\
% --- & All questions carry the same weight for the final mark.\\
% --- & For all questions, draw a schematic and list values and equations used.
%       Looked up properties should be listed with the respective source (table,
%       graph, etc) and temperature.\\
\end{tabular}
\end{footnotesize}

\hrulefill

%%%%%%%%%%%%%%%%%%%%%%%%%%%%%%%%%%%%
%   QUESTIONS
%%%%%%%%%%%%%%%%%%%%%%%%%%%%%%%%%%%%

\begin{questions}

\fullwidth{\Large \textbf{Simple Questions}}

% Multiple Choice question with Latex and one exact solution
\question
For a solution The solution of $Ax=b$ to exist $b$ must be perpendicular to:
\begin{choices}
    \choice Row space
    \choice Null Space
    \choice Column Space - also called range
    \CorrectChoice Left Null Space
\end{choices}

% Multiple Choice question with Latex and multiple exact solutions
\question
Which methods can be used to solve linear Ordinary Differential Equation (ODE) of nth order with constant coefficients:
\begin{choices}
    \choice Integrating factor
    \CorrectChoice Method of Undetermined Coefficients
    \CorrectChoice Variation in Parameters
    \CorrectChoice Laplace Transform
\end{choices}

% Question with Latex and  integer solution
\question Find the solution of $y=\int_0^2 \frac{x^3}{2} dx$

\fullwidth{\begin{solution}
$y=2$ \end{solution}}

% Question with Latex and numerical solution
\question Find the solution of $y=\sqrt{2}$

\fullwidth{\begin{solution}
$y=1.414214$\\
Can we make the correct answer between $1.413 \leq y \geq 1.415$ for moodle?
 \end{solution}}

\fullwidth{\Large \textbf{Questions with variables}}

% Question with Latex a numerical solution but the question has a variable
\question Find the solution of $y=\sqrt{2}$\\
- I would like the 2 to be a variable from a list in Moodle.\\
- ie let x = [ 2 3 4 5]\\
- the solution calculated and a tolerance of say $y = \sqrt{x} \pm 0.001$

\fullwidth{\begin{solution}
$y=1.414214$\\
Can we make the correct answer between $1.413 \leq y \geq 1.415$ for moodle?
 \end{solution}}

\end{questions}
\end{document}

% Question with symbolic solution - use sympy?
\question
Show that the solution of $ \frac{d^2y}{dx^2} + \frac{dy}{dx} - 2y = 2x+40\cos(2x)$  is:
$y(x) = c_1e^x +c_2e^{-2x} -\frac{1}{2} - x + 6\cos(2x) - 2\sin(2x)$

\fullwidth{\begin{solution}
Can use method of undetermined coefficients.\\
Can also use symbolic - take Laplace transform and invert:\\
a) In Matlab \\
 syms s x, f= ilaplace$(( (2/s^2)- (40*s)/(s^2+4) )/(s^2+s-2),x)$\\
b) In Maple\\
with(inttrans): , f=invlaplace$((2/s^2-40*s/(s^2+4))/(s^2+s-2), s, x)$\\
both have the solution:
\[f=-x-7/2\,{{\rm e}^{-2\,x}}+6\,\cos \left( 2\,x \right) -2\,\sin
 \left( 2\,x \right) -2\,{{\rm e}^{x}}-1/2\]
\end{solution}}

% Another Question with symbolic solution - use sympy?
\question
Show that the particular solution of $ \frac{d^2y}{dx^2} + y = \sec(x)\tan(x)$  is:
$y(x) = x\cos(x) - \sin(x) + \sin(x)\ln|\sec(x)|$

\fullwidth{\begin{solution}
Can use method of undetermined coefficients.\\
Can also use symbolic. In Maple\\
ode1 := diff(y(x), x, x)+y(x) = sec(x)*tan(x)\\
dsolve(ode1)\\
both have the solution -- equation below from Maple command latex(dsolve(ode1)):
\[y \left( x \right) =\sin \left( x \right) {\it \_C2}+\cos \left( x
 \right) {\it \_C1}-\ln  \left( \cos \left( x \right)  \right) \sin
 \left( x \right) -\sin \left( x \right) +\cos \left( x \right) x \]

\end{solution}}

% Another Question with symbolic solution and variables - use sympy?
\question
Show that the  solution of $ \left( \frac{dy}{dx}\right)^2 -4y+ 4 = 0$  is:
$ \pm \sqrt{y-1} = x$ when $y\neq1$\\
\\ I would like to choose the coefficients  of the first equation from a list and symbolically solve

\fullwidth{\begin{solution}
Equation is non-linear but separable \\
Can also use symbolic. In Maple\\
ode2 := $ \left( \frac{dy}{dx}\right)^2 -4y+ 4 = 0$\\
dsolve(ode2)\\
\end{solution}}



\end{questions}
%

\end{document}
